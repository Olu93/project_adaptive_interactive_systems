\documentclass[12pt,a4paper,oneside]{article}

\title{RouteToDam: Provides information on day trips to Rotterdam for students at Utrecht University}
\date{\today}
\author{Soring Dragan, Evangelia Giannikou, \\Mikhail Ternyuk, Olusanmi Hundogan}
\pagenumbering{gobble}
\usepackage{eurosym}
\usepackage{hyperref}
\usepackage{amsmath}
\usepackage{amssymb}
\usepackage{tabularx,booktabs}
\usepackage{multicol}
\usepackage{array}
\usepackage{float}
\usepackage[english]{babel}
\usepackage{quoting}
\usepackage{csquotes}
\usepackage{enumitem}
\usepackage{natbib}
\usepackage{xcolor}
\usepackage{subfiles}
\usepackage{ragged2e}
\newcolumntype{M}[1]{>{\centering\arraybackslash}m{#1}}
\usepackage[backend=biber, sorting=none]{biblatex}
\addbibresource{references.bib}

\begin{document}

\maketitle

In order to create a recommendation system for traveling it is important to first define it, as well as the needs for adaptation and constraints. RouteToDam will be modeled as a \textbf{mobile application} providing recommendations on day trips based on the users’ profiles, needs, and goals. The reason for selecting to model a system for mobiles lies in the fact that nowadays mobile devices are primarily used for information access. E-tourism \cite{mobile_recommendation_systems} has given its place to mobile tourism since it offers new opportunities that would not be possible if the system was modeled as a website. For instance, the knowledge of the exact user location offers the possibility to give recommendations of points of interest while the user is moving from one place to another. Furthermore, the system can know the user’s mobility history and thus provide better recommendations for future locations.\\

Another advantage of the mobile recommendations is that they increase the usability of mobile tourism applications as they provide personalized and more focused content. In particular, the focus group in our system refers to students from the University of Utrecht. As research has shown \cite{rita2019millennials}, the most important motivations for young people to travel are the need to experience a different lifestyle, to relax, and the desire to escape from the ordinary. Other groups are also interested in sightseeing and attending local events. Keeping that in mind, it is important to make sure that RouteToDam will adapt to all these needs and be able to fulfill every student’s expectations.\\

A tourism recommender system like that should also adapt to its city. Rotterdam’s tourism gave a lot of focus on the culture \cite{rotterdam}. The result of that was the image of a modern art city with futuristic architecture. This shifted away from the traditional Dutch model leading to the enrichment of the cultural life and profile of the city. These elements are what makes this city different from the other ones in the Netherlands and our system should be able to adapt to that too. \\

Why do we need user adaptation? The satisfaction level of each student of the group, as well as the number of visited locations and sights in one day, depends on how well the system adapts to users preferences. First of all, the system should provide the most related information about the city (city history, exclusive places, events and etc.), but at the same time, it suppose to be suitable for each student in a group. As a result, the application should provide a city tour plan for the whole group, but satisfy the requirements of each personality. Also, the plan should be limited to one day. Since we are considering a group of students, the system must adapt to their preferences, which are may vary. Moreover, each student could have a different experience of visiting Rotterdam. For instance: if a tourist visiting the city for the first time, she will probably want to see as many places as she can and especially the most popular of them, while a student who already been in Rotterdam, probably prefer more specific places or similar to the liked places in the past \cite{predicting}. Students may have different financial support and we suppose the system has to adapt the final result according to the budget of the group members.\\

The system will adapt to the user's experience in the process of selecting locations or items she has already visited before or especially remembered during the last visit. Thus, the application can offer events, locations, and specific places with a detailed description on a similar topic. At the same time, when a student visits Rotterdam for the first time, she needs help to choose the direction to get to know the city, and the system firstly will offer the most iconic and popular places in the city to users.\\

However, every person is different in their preferences, especially young students. The application should provide the user ability to indicate the personal interests of the time spending. For example clubbing, city tours, visiting special events or funds and etc. In this case, the system will take into account not only popular places, according to user requirements and initial rating, but also the opening hours of item (locations), the time of events in order to distribute visiting times optimally for users of different interests. For example, the first priority for a student is to get to the show in Rotterdam, but it starts in the evening. Before the show, the student also wants to see the main locations and sights of the city, we suppose that our system should take into account the interests of the user and adapt the tour plan to the opening timing of the event. \\

When we planning a program for visiting museums or certain events, bars, and clubs, the issue of budget is very important. Nowadays, museum tickets, and unique events, even with student discounts can be quite expensive. We assume that the system may ask the students about the budget they suppose to spend on a touristic program. By identified interests and existing knowledge about the city, the application should adapt the list of city locations and sights to associated financial costs. The budget for the tour program will be equal to the minimum budget of all group members. \\


\indent As another type of adaptation we consider the described system to be able to adapt to changes in context/environment. Here, we consider multiple subsets of context adaptation. Firstly, the system will adapt to the \textbf{location}. It will suggest popular sights that are located within a formerly specified range from the current location of the group. This suggestion will take place as notifications which the users can ignore if they are not interested anymore. Secondly, the system will adapt to the \textbf{weather}. As Rotterdam is subject to unpredictable rains \cite{creemers2015meteorological}, the system will adapt to the changes in the weather by switching from outdoor activities to indoor activities. Thirdly, the system will take the \textbf{time} of the day into account. In this case, we differentiate between morning, daytime and nighttime activities. The morning might be the perfect time for a museum, while the night might imply going dancing, for example. Lastly, the system will consider the specific \textbf{date} when the trip is planned and use it to recommend unique events that might interesting for the visiting group.

\indent The system will consist of two major types of adaptation. On one side, we have user adaptation and then we went for beers!!! 

\printbibliography
\newpage
\end{document}